\twocolumn
этого шара плоскостью $\mathit{ADB}$ (рис. 3). Так как $\measuredangle\ \mathit{AOB}$ прямой, хорда $\mathit{AB}$ - диаметр окружности с центром в точке $\mathit{O_1}$. Через точку $\mathit{O_1}$, перпендикулярно $\mathit{AB}$ проведем плоскость $\Pi$, она пройдет через центр шара и будет перпендикулярна плоскостям $\mathit{ACB}$ и $\mathit{ADB}$. Пусть $\mathit{E}$ и $\mathit{F}$ - точки пересечения плоскости $\Pi$ c окружностями сечений $\mathit{ACB}$ и $\mathit{ADB}$ соответственно Очевидно, $\measuredangle\mathit{EO_1F}$ является линейным углом искомого двугранного угла между плоскостями $\mathit{ACB}$и $\mathit{ADB}$. Так как $\mathit{OO_2}\bot\mathit{FO_1}$ и $OO_1\bot EO_1$, то $\measuredangle\ EO_1F\ =\ \measuredangle\ O_2OO_1$. Стороны треугольника $O_2OO_1$ легко найти: так как $\measuredangle\ AO_2O_1\ =$
\begin{displaymath}
	= \measuredangle\ AFB = \measuredangle\ ADB = 60\degree, то O_2O_1 = \frac{R\sqrt{3}}{2};
\end{displaymath}
очевидно, $OO_1 = \frac{R\sqrt{3}}{2}$. Тогда $\sin{\measuredangle\ O_2OO_1} = \frac{O_2O_1}{OO_1}$


\begin{enums}
	\item[\textbf{3.}] При $a \leqslant\ - 1$ наименьшим корнем будет $a + 1$; при $- 1\leqslant a\leqslant 1$ наименьшим корнем будет $-2a$. У к а з а н и е. Корнями данного уравнения являются $x_1 = -2a, x_2 = a + 1, x_3 = -a - 1$, Значения параметра $a$, при которых наименьшим корнем будет $x_3$, найдутся из системы неравенств\\
	\begin{equation*}
		\begin{cases}
			x_3\leqslant x_1,\\
			x_3\leqslant x_2, 
		\end{cases}
		\text{т. е.}  
		\begin{cases}
			- a - 1\leqslant - 2a,\\
			- a - 1\leqslant a + 1,
		\end{cases}
	\end{equation*}
	откуда $-1\leqslant a\leqslant 1$. Аналогично можно найти значения параметра $a$, при которых наименьшим корнем будет $x_2$, а затем $x_1$.
	
	\item[\textbf{4.}]
	\begin{math}
		x = ( - 1)^n \frac{\pi}{6} + n\pi, y = ( - 1)^k \times \arcsin{\frac{3}{4}} + k\pi, \text{где $n$ и $k$ - целые числа}
	\end{math}
\end{enums}

\subsubsection*{В а р и а н т 4}

\begin{enums}
	\item $20\leqslant x\leq 60$. У к а з а н и е. Пусть $x$(\textit{км/час})-первоначальная скорость велосипедиста. Из условий задачи вытекает неравенство
	\begin{equation*}
		\frac{60}{x} \leqslant\ 1 + \frac{1}{3}\ + \frac{60 - x}{x + 4}.
	\end{equation*}
	Для получения ответа надо еще учесть, что $0\leq\  x\leq\  60$.
	
	\item Радиус большего сечения равен\\
	\begin{math}
		R\sqrt{\frac{10 + 4\sqrt{2}}{17}}.
	\end{math}
	У к а з а н и е. Через точку касания прямой с шаром провести плоскость, перпендикулярную касательной.
	Сечение шара этой плоскость есть большой круг, он пересекается с упомянутыми в условии задачи сечениями по их диаметрам
	\newpage
	
	Сечение шара этой плоскость есть большой круг, он пересекается с упомянутыми в условии задачи сечениями по их диаметрам
	\item $-2\leq x\leqslant-1$
	
	\item При всех значениях параметра $ \mathit{a} $ уравнение имеет серию корней $x=\frac{\pi}{4} + \pi n$, где $ \mathit{n} $ - целое число
\end{enums}
\subsection*{Ф и з и к а}
\subsubsection*{Математико-механический факультет}
\begin{enums}
	\item \begin{math}
		\cos{\alpha}\ = 1 - \frac{m^2v^2}{2gL(m+M)^2}.
	\end{math}
	
	\item $M_\text{атм}\approx 52\cdot 10^{17} \textit{кг}$
	
	\item \begin{math}
		I = \frac{F_v}{U} = 2\cdot 10^3a=2\textit{ка}.
	\end{math}
	
	\item \begin{math}
		n_2 = n_1 \frac{\sin{a}}{\sin{90-a}}\approx1,4.
	\end{math}
\end{enums}

\subsubsection*{Физический факультет}

\begin{enums}
	\item $Q = 94,5\textit{дж}$
	\item \begin{math}
		m_\textit{в} = \rho_\textit{в} (1 - \frac{p_1}{p_a + p_\textit{в}gh})\approx 300 \textit{г.}
	\end{math}
	\item Cм. рис. \ref{Ris1}
	\item $l = 2h$
\end{enums}

\subsubsection*{К статье "Московский электротехнический институт связи"}
\begin{center}
	\textit{(см. "Квант" № 5)}
\end{center}

\subsection*{М а т е м а т и к а}
\subsubsection*{Факультет автоматики, телемеханики и электроники}

\begin{enums}
	\item \begin{math}
		x_1 = - \frac{\pi}{6} + \frac{\pi (2n + 1)}{2}, x_2 = \frac{\pi}{12} + \frac{\pi n}{2}
	\end{math}
	(\textit{n} - целое). У к а з а н и е. Уравнение приводится к виду \begin{math}
		\sin{3x} = \sin{(\frac{\pi}{3} - x)}
	\end{math}
	\item $x=1$. У к а з а н и е. Привести уравнение к виду $(2^x - \frac{2}{2^x})^3 = 1$
\end{enums}
\begin{figure}[h]
	\centering
	\includegraphics[width=0.6\linewidth]{квант желтая}
	\vspace*{0pt}
	\caption{}\label{Ris1}
\end{figure}
\onecolumn